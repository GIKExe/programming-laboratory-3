
Схема программы: \\
\includegraphics[width=17cm]{1.pdf} \\
\newpage
\includegraphics[width=17cm]{2.pdf} \\
\newpage
\includegraphics[width=17cm]{3.pdf} \\
\newpage

Код программы на C: \\
% \inputminted{C}{main.c}
\begin{minted}{C}
#include <math.h>
#include <stdio.h>

// Функция для вычисления значения кривой в точке x
double f(double x) {
  double res = 2 * pow(x, 3) + pow(x, 2) + 1;
  if (res<0) res=0; // отбрасывание отрицательного
  return res;
}

// Функция вычисления интеграла методом Симпсона
double integral(double a, double b, int n) {
  double h = (b - a) / n;   // шаг
  double sum = f(a) + f(b); // сумма концов

  for (int i = 1; i < n; i++) {
    double x = a + i * h;
    if (i % 2 == 0) sum += 2 * f(x);
    else sum += 4 * f(x);
  } return sum * h / 3;
}

// Процедура для оценки погрешности методом Рунге
void estimate_error(double a, double b, int n, double* error) {
  double integral_n = integral(a, b, n);
  double integral_2n = integral(a, b, n * 2);
  *error = fabs(integral_2n - integral_n);
}

void clear_buffer() {
  while (getchar() != '\n');
}

int main() {
  double a, b, result, error;
  int ls = 0, ns = 0, is = 0, es = 0, x, n, choice;
  do {
    printf("\nМеню:");
    printf("\n1. Ввести пределы (a -> b)");
    if (ls) printf(" (%f, %f)", a, b);

    printf("\n2. Ввести кол-во разбиений (n)");
    if (ns) printf(" (%d)", n);

    printf("\n3. Вычислить интеграл");
    if (is) printf(" (%.6f)", result);

    printf("\n4. Оценить погрешность");
    if (es) {
      double procent = (1 - fabs(result - error) / result) * 100;
      printf(" (%.6f, %.6f%%)", error, procent);
    }

    printf("\n5. Выход");
    printf("\nВыберите опцию: ");
    scanf("%d", &choice);

    
    switch (choice) {
      case 1:
        printf("\nВведите пределы: ");
        ls = 0;
        if (scanf("%lf %lf", &a, &b) == 2) {
          if (a < 0) {a = 0; printf("предел А был изменён на 0\n");}
          if (b >= a) ls = 1;
          else printf("Неверный ввод: B должно быть >= A\n");
        } else {
          printf("Ошибка ввода\n");
          clear_buffer();
        }; break;

      case 2:
        printf("\nВведите кол-во разбиений: ");
        ns = 0;
        if (scanf("%d", &n) == 1) {
          if (n > 0 && n % 2 == 0) ns = 1;
          else printf("Неверный ввод: n должно быть >0 и чётным\n");
        } else {
          printf("Ошибка ввода\n");
          clear_buffer();
        }; break;

      case 3:
        if (ls && ns) {
          result = integral(a, b, n);
          is = 1;
        } else printf("\nСначала Введите данные!\n");
        break;

      case 4:
        if (ls && ns) {
          estimate_error(a, b, n, &error);
          es = 1;
          n *= 2; // Увеличиваем n для следующего вычисления
        } else printf("\nСначала Введите данные!\n");
        break;

      case 5:
        printf("\nВыход...\n");
        break;

      default:
        printf("\nНеверный ввод! Выберите пункт от 1 до 5\n");
        clear_buffer();
    }
  } while (choice != 5);
  return 0;
}
\end{minted}

\newpage

Примеры работы программы: \\ \\
\includegraphics[width=17cm]{p1.png} \\
\includegraphics[width=17cm]{p2.png} \\
\includegraphics[width=17cm]{p3.png} \\