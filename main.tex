\documentclass[oneside,a4paper,14pt]{extarticle} %размер шрифта 14
\usepackage[T1,T2A]{fontenc}
\usepackage[
	a4paper,
	letterpaper,
	top=2cm,
	bottom=2cm,
	left=2.5cm,
	right=1.5cm
]{geometry}
\usepackage[utf8]{inputenc} %кодировка текста
\usepackage[russian]{babel} %поддержка русского языка
\usepackage{textcomp} %текстовые символы
\usepackage{indentfirst} %корректировка отступов
\usepackage{listings} %для кода !!!
\usepackage{minted} %для кода !!!
\usepackage{enumitem} %для нумерации буквами
\usepackage{graphicx} %работа с изображениям
\usepackage{mwe} % for blindtext and example-image-a in example
\usepackage{wrapfig}
\usepackage{caption}
\usepackage{amsmath}  % для формул и символов
\usepackage{amsfonts}
\usepackage{amsthm}
\usepackage{graphicx}
\usepackage[all]{xy}
\usepackage[breaklinks]{hyperref}
%%размеркегляузаголоковразделов
\usepackage{titlesec}
\titleformat{\section}
{\normalsize\bfseries}
{\thesection} {1em}{}
\titleformat{\subsection}
{\normalsize\bfseries}
{\thesubsection} {1em}{}
\titleformat{\subsubsection}
{\normalsize\bfseries}
{\thesubsection} {1em}{}

\renewcommand\baselinestretch{1.33}\normalsize % межстрочный интервал
\setlength{\parindent}{1.25cm}
\usepackage{indentfirst}


\begin{document}
	\newpage\thispagestyle{empty}
	\begin{center}
		МИНИСТЕРСТВО НАУКИ И ВЫСШЕГО ОБРАЗОВАНИЯ\\
			РОССИЙСКОЙ ФЕДЕРАЦИИ
			ФЕДЕРАЛЬНОЕ ГОСУДАРСТВЕННОЕ БЮДЖЕТНОЕ\\
			ОБРАЗОВАТЕЛЬНОЕ
			УЧРЕЖДЕНИЕ ВЫСШЕГО ОБРАЗОВАНИЯ\\
			«ВЯТСКИЙ ГОСУДАРСТВЕННЫЙ УНИВЕРСИТЕТ»\\
			Институт математики и информационных систем\\
			Факультет автоматики и вычислительной техники\\
			Кафедра электронных вычислительных машин
	\end{center}
	\vspace{20mm}

	\begin{center}
		Отчёт по лабораторной работе №3\\
		по дисциплине\\
		<<Программирование>>\\
		% << >>\\
	\end{center}
	\vspace{48mm}

	Выполнил студент гр. ИВТб-1303-06-00 \hspace{10mm} \rule[-0,5mm]{23mm}{0.15mm}\,/Гортоломей И.К./
	Проверил преподаватель кафедры ЭВМ \hfill  \rule[-0,5mm]{30mm}{0.15mm}\,/Баташев П.А./

	\vfill
	\begin{center}
		Киров\\
		2025
	\end{center}

	\newpage%\thispagestyle{empty}

	\section*{Цель} Цель работы: освоить синтаксис построения процедур и функций, изучить способы передачи данных в подпрограммы, получить навыки организации минимального пользовательского интерфейса.

	\section*{Задание}
	\begin{enumerate}
\item Реализовать программу вычисления площади фигуры, ограниченной кривой и осью OX (в положительной части по оси OY).
\item Вычисление определенного интеграла должно выполняться численно, с применением метода.
\item Пределы интегрирования вводятся пользователем.
\item Взаимодействие с пользователем должно осуществляться посредством case-меню.
\item Требуется реализовать возможность оценки погрешности полученного результата.
\item Необходимо использовать процедуры и функции там, где это целесообразно (программа
должна содержать минимум одну процедуру, одну функцию, один пример передачи данных в
подпрограммы по ссылке, один пример передачи данных в подпрограммы по значению).
\end{enumerate}

	\section*{Дано:}
	\begin{itemize}
		\item Уравнение кривой: $2*x^3+1*x^2+0*x+1$
		\item Метод: Симпсона
		\item Язык: Си
	\end{itemize}
	\newpage

	\section*{Решение}
  
Код программы на C: \\
% \inputminted{C}{main.c}
\begin{minted}{C}
#include <stdio.h>
#include <math.h>
#include <stdlib.h>

// Функция для очистки экрана
void clear(void) {
#ifdef _WIN32
  system("cls");
#else
  system("clear");
#endif
}

// Функция для вычисления значения кривой в точке x
double f(double x) {
  return 2 * pow(x, 3) + pow(x, 2) + 1;
}

// Функция вычисления интеграла методом Симпсона
double simpson_integral(double a, double b, int n) {
  double h = (b - a) / n;
  double sum = f(a) + f(b);

  for (int i = 1; i < n; i++) {
    double x = a + i * h;
    if (i % 2 == 0)
      sum += 2 * f(x);
    else
      sum += 4 * f(x);
  }
  return sum * h / 3;
}

// Процедура для оценки погрешности методом Рунге (передача n по ссылке)
void estimate_error(double a, double b, int* n, double* error) {
  double integral_n = simpson_integral(a, b, *n);
  double integral_2n = simpson_integral(a, b, *n * 2);
  *error = fabs(integral_2n - integral_n) / 15;
}

int main() {
  double a = 0, b = 0, result, error;
  int n = 0, choice;
  int limits_set = 0, n_set = 0, integral_set = 0, error_set = 0;

  do {
    printf("\nМеню:\n");
    printf("1. Ввести пределы интегрирования (a, b)");
    limits_set ? printf(" (%f, %f)\n", a, b) : printf("\n");
    printf("2. Ввести количество разбиений n");
    n_set ? printf(" (%d)\n", n) : printf("\n");
    if (limits_set && n_set) {
      printf("3. Вычислить интеграл");
      integral_set ? printf(" (%.6f)\n", result) : printf("\n");
      printf("4. Оценить погрешность");
      error_set ? printf(" (%.6f)\n", error) : printf("\n");
      printf("5. Выход\n");
    } else {
      printf("3. [Недоступно - введите данные]\n");
      printf("4. [Недоступно - введите данные]\n");
      printf("5. Выход\n");
    }
    printf("Выберите опцию: ");
    scanf("%d", &choice);
    clear();

    switch (choice) {
      case 1:
        printf("Введите пределы интегрирования (a b): ");
        if (scanf("%lf %lf", &a, &b) == 2) {
          limits_set = 1;
          // printf("Пределы установлены: a=%.2f, b=%.2f\n", a, b);
        } else {
          printf("Ошибка ввода!\n");
          while (getchar() != '\n'); // Очистка буфера
        }
        break;

      case 2:
        printf("Введите количество разбиений n: ");
        if (scanf("%d", &n) == 1 && n > 0) {
          n_set = 1;
        } else {
          printf("Ошибка ввода! n должно быть положительным числом.\n");
          while (getchar() != '\n'); // Очистка буфера
        }
        break;

      case 3:
        if (limits_set && n_set) {
          result = simpson_integral(a, b, n);
          integral_set = 1;
        } else printf("Сначала введите данные!\n");
        break;

      case 4:
        if (limits_set && n_set) {
          estimate_error(a, b, &n, &error);
          error_set = 1;
          n *= 2; // Увеличиваем n для следующего вычисления
        } else printf("Сначала введите данные!\n");
        break;

      case 5:
        printf("Выход...\n");
        break;

      default:
        printf("Неверный ввод! Выберите пункт от 1 до 5.\n");
    }
  } while (choice != 5);
  return 0;
}
\end{minted}

\newpage

Примеры работы программы: \\ \\
\includegraphics[width=17cm]{p1.png} \\
\includegraphics[width=17cm]{p2.png} \\
\includegraphics[width=17cm]{p3.png} \\
	\newpage

	% \section*{Выводы}

\end{document}